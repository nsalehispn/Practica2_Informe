\documentclass[spanish,a4paper,11pt,twoside]{report}

%%%%%%%%%%%%%%%%%%%%%%%%%%%%%%%%%%%%%%%%%%%%%%%%%%%%%%%%%%%%%%%%%%%%%%%%%%%%%%%
\usepackage[dvips]{graphicx}
\usepackage[dvips]{epsfig}
\usepackage[utf8]{inputenc}
\usepackage[english]{babel}
\usepackage{alltt}
\usepackage{templates/algorithm}
\usepackage{templates/algorithmic}
\usepackage{templates/multirow}

%%%%%%%%%%%%%%%%%%%%%%%%%%%%%%%%%%%%%%%%%%%%%%%%%%%%%%%%%%%%%%%%%%%%%%%%%%%%%%%

\newcommand{\SONY}{{\sc Sony}}
\newcommand{\MICROSOFT}{{\sc Microsoft}}
\newcommand{\GCC}{\textsf{\textsc{G}CC}}
\newcommand{\INTEL}{\textsf{\textsc{I}ntel}}

%%% Traducimos el pseudocodigo
\renewcommand{\algorithmicwhile}{\textbf{mientras}}
\renewcommand{\algorithmicend}{\textbf{fin}}
\renewcommand{\algorithmicdo}{\textbf{hacer}}
\renewcommand{\algorithmicif}{\textbf{si}}
\renewcommand{\algorithmicthen}{\textbf{entonces}}
\renewcommand{\algorithmicrepeat}{\textbf{repetir}}
\renewcommand{\algorithmicuntil}{\textbf{hasta que}}
\renewcommand{\algorithmicelse}{\textbf{en otro caso}}
\renewcommand{\algorithmicfor}{\textbf{para}}

%\newcommand{\RETURN}{\textbf{retornar} }
\newcommand{\RET}{\STATE \textbf{retornar} }
\newcommand{\TO}{\textbf{hasta} }
\newcommand{\AND}{\textbf{y} }
\newcommand{\OR}{\textbf{o} }

%%%%%%%%%%%%%%%%% Creamos un entorno para listar código fuente %%%%%%%%%%%%%%%
\newenvironment{sourcecode}
{\begin{list}{}{\setlength{\leftmargin}{1em}}\item\scriptsize\bfseries}
{\end{list}}

\newenvironment{littlesourcecode}
{\begin{list}{}{\setlength{\leftmargin}{1em}}\item\tiny\bfseries}
{\end{list}}

\newenvironment{summary}
{\par\noindent\begin{center}\textbf{Abstract}\end{center}\begin{itshape}\par\noindent}
{\end{itshape}}

\newenvironment{keywords}
{\begin{list}{}{\setlength{\leftmargin}{1em}}\item[\hskip\labelsep \bfseries Keywords:]}
{\end{list}}

\newenvironment{palabrasClave}
{\begin{list}{}{\setlength{\leftmargin}{1em}}\item[\hskip\labelsep \bfseries Palabras clave:]}
{\end{list}}


%%%%%%%%%%%%%%%%%%%%%%%%%%%%%%%%%%%%%%%%%%%%%%%%%%%%%%%%%%%%%%%%%%%%%%%%%%%%%%%
% Format
%%%%%%%%%%%%%%%%%%%%%%%%%%%%%%%%%%%%%%%%%%%%%%%%%%%%%%%%%%%%%%%%%%%%%%%%%%%%%%%

%%\topmargin -4 mm
%\topmargin -21 mm
%\headheight 10 mm
%\headsep 10 mm

%\textheight 229 mm
%\textheight 246 mm

%\oddsidemargin -5.4 mm
%\evensidemargin -5.4 mm
\oddsidemargin 5 mm
\evensidemargin 5 mm

%\oddsidemargin -3 mm
%\evensidemargin -3 mm

%\textwidth 17 cm
\textwidth 15 cm
%\columnsep 10 mm

\input{amssym.def}

%%%%%%%%%%%%%%%%%%%%%%%%%%%%%%%%%%%%%%%%%%%%%%%%%%%%%%%%%%%%%%%%%%%%%%%%%%%%%%%

\begin{document}

%%%%%%%%%%%%%%%%%%%%%%%%%%%%%%%%%%%%%%%%%%%%%%%%%%%%%%%%%%%%%%%%%%%%%%%%%%%%%%%
% First Page 
%%%%%%%%%%%%%%%%%%%%%%%%%%%%%%%%%%%%%%%%%%%%%%%%%%%%%%%%%%%%%%%%%%%%%%%%%%%%%%%

\pagestyle{empty}
\thispagestyle{empty}


\newcommand{\HRule}{\rule{\linewidth}{1mm}}
\setlength{\parindent}{0mm}
\setlength{\parskip}{0mm}
\vspace*{\stretch{1}}

\begin{center}
\includegraphics[width=0.2\textwidth]{images/logotipo-secundario-ULL}\\[0.25cm]
\end{center}

\HRule
\begin{center}
        {\Huge Proyecto de la asignatura Computacion Avanzada } \\[2.5mm] 
        {\Huge Título del Artículo :  A parallel tabu search algorithm for solving
the container loading problem
} \\[5mm]

        {\Large Autor : ROBABEH SALEHI} \\[10mm]

        {\em Computación Avanzada} \\[5mm]
        Lenguajes y Sistemas Informáticos \\[5mm]
        Escuela Técnica Superior de Ingeniería y Tecnología \\[5mm]
        
        Universidad de La Laguna \\
\end{center}
\HRule
\vspace*{\stretch{2}}
\begin{center}
  La Laguna, \today 
\end{center}

%%%%%%%%%%%%%%%%%%%%%%%%%%%%%%%%%%%%%%%%%%%%%%%%%%%%%%%%%%%%%%%%%%%%%%%%%%%%%%%
\begin{abstract}
{\em

%El objetivo de este trabajo ha sido ....
%
%bla, bla, bla
%
%bla, bla, bla
%
%bla, bla, bla
Abstract
This paper presents a parallel tabu search algorithm for the container loading problem with
a single container to be loaded. The emphasis is on the case of a weakly heterogeneous load.
The distributed-parallel approach is based on the concept of multi-search threads according to
Toulouse et al. [Issues in designing parallel and distributed search algorithms for discrete optimization
problems, Publication CRT-96-36, Centre de recherche sur les transports, Universitede
Montreal, Canada, 1996] i.e., several search paths are investigated concurrently. The
parallel searches are carried out by differently configured instances of a tabu search algorithm,
which cooperate by the exchange of (best) solutions at the end of defined search phases. The
parallel search processes are executed on a corresponding number of LAN workstations. The
efficiency of the parallel tabu search algorithm is demonstrated by an extensive comparative
test including well-known reference problems and loading procedures from other authors.
 2003 Elsevier Science B.V. All rights reserved.



In the interest of stability of the load, both horizontal dimensions of each box are
to be supported according to a predefined percentage. In any case the centre of gravity
of each box must be supported in order to avoid boxes tipping over. It is assumed
that the centre of gravity and the geometric centre of each box coincide.
Box types are defined as follows. Two boxes are the same type if they coincide in
all three side dimensions. On the basis of this concept of box types, the following
three categories of box sets can be distinguished. A homogeneous box set is given
if all boxes are of the same type. A box set is called weakly heterogeneous if there
exist a few box types and many items per type. Finally, a strongly heterogeneous
box set is characterized by a greater number of box types and only few items per
type. Here, a weakly heterogeneous set of boxes is assumed.
In the recent years, many (sequential) solution methods for the container loading
problem have been developed. It is well known that the container loading problem is
NP-hard (cf. [18]). Hence, the methods developed are heuristic approaches.

The only block of a 1-arrangement is always placed in the reference corner of the
packing space. From the two blocks of a 2-arrangement, one is arranged in the
reference corner. The second block can alternatively be placed as a neighbour in
x-direction (arrangement type ‘‘in front of’’), as a neighbour in y-direction (arrangement
type ‘‘beside’’) or as a neighbour in z-direction (arrangement type ‘‘above’’). In
the case of a placement according to arrangement type ‘‘in front of’’, the block with
the larger y-dimension is positioned in the reference corner, while for the arrangement
type ‘‘beside’’ the block with the larger x-dimension is positioned in the reference
corner. The arrangement type ‘‘above’’ is only used if both horizontal
dimensions of a block are not smaller than the corresponding dimensions of the
other block. The block with the larger horizontal dimensions is positioned in the reference
corner and the other above. Fig. 2 illustrates a 1-arrangement and two 2-
arrangements of the arrangement types ‘‘in front of’’ and ‘‘beside’’.

}

\begin{palabrasClave}
Tabu searchin: is one of the searching method.
\end{palabrasClave}

\end{abstract}
%%%%%%%%%%%%%%%%%%%%%%%%%%%%%%%%%%%%%%%%%%%%%%%%%%%%%%%%%%%%%%%%%%%%%%%%%%%%%%%

%%%%%%%%%%%%%%%%%%%%%%%%%%%%%%%%%%%%%%%%%%%%%%%%%%%%%%%%%%%%%%%%%%%%%%%%%%%%%%%
%\newpage{\pagestyle{empty}\cleardoublepage}

\pagestyle{myheadings} %my head defined by markboth or markright
% No funciona bien \markboth sin "twoside" en \documentclass, pero al
% ponerlo se dan un mont�n de errores de underfull \vbox, con lo que no se
% ha puesto.
\markboth{ROBABEH SALEHI}{A parallel tabu search algorithm for solving
the container loading problem}

%%%%%%%%%%%%%%%%%%%%%%%%%%%%%%%%%%%%%%%%%%%%%%%%%%%%%%%%%%%%%%%%%%%%%%%%%%%%%%%
%Numeracion en romanos
\renewcommand{\thepage}{\roman{page}}
\setcounter{page}{1}


%\tableofcontents

%%%%%%%%%%%%%%%%%%%%%%%%%%%%%%%%%%%%%%%%%%%%%%%%%%%%%%%%%%%%%%%%%%%%%%%%%%%%%%%

%\listoffigures

%%%%%%%%%%%%%%%%%%%%%%%%%%%%%%%%%%%%%%%%%%%%%%%%%%%%%%%%%%%%%%%%%%%%%%%%%%%%%%%

%\listoftables

%%%%%%%%%%%%%%%%%%%%%%%%%%%%%%%%%%%%%%%%%%%%%%%%%%%%%%%%%%%%%%%%%%%%%%%%%%%%%%%
%\newpage{\pagestyle{empty}\cleardoublepage}

%%%%%%%%%%%%%%%%%%%%%%%%%%%%%%%%%%%%%%%%%%%%%%%%%%%%%%%%%%%%%%%%%%%%%%%%%%%%%%%
%Numeracion a partir del capitulo I
\renewcommand{\thepage}{\arabic{page}}
\setcounter{page}{1}

\setlength{\parindent}{5mm}

%%%%%%%%%%%%%%%%%%%%%%%%%%%%%%%%%%%%%%%%%%%%%%%%%%%%%%%%%%%%%%%%%%%%%%%%%%%%%%%
\chapter{Heuristic Algorithm}
\label{chapter:obj}

%%%%%%%%%%%%%%%%%%%%%%%%%%%%%%%%%%%%%%%%%%%%%%%%%%%%%%%%%%%%%%%%%%%%%%%%%%%%%
% Chapter 1: Motivación y Objetivos 
%%%%%%%%%%%%%%%%%%%%%%%%%%%%%%%%%%%%%%%%%%%%%%%%%%%%%%%%%%%%%%%%%%%%%%%%%%%%%%%

Los objetivos le dan al lector las razones por las que se realizó el
proyecto o trabajo de investigación.

%---------------------------------------------------------------------------------
\section{Sección Uno}
\label{1:sec:1}
  Primer párrafo de la primera sección.


%---------------------------------------------------------------------------------
\section{Sección Dos}
\label{1:sec:2}
  Primer párrafo de la segunda sección.

\begin{itemize}
  \item Item 1
  \item Item 2
  \item Item 3
\end{itemize}


%""""""""""""""""""""""""""""""""""""""""""""""""""""""""""""""""""""""""""""""
\begin{center}
\includegraphics[width=1\textwidth]{images/picn5.eps}\\[0.25cm]
\end{center}

%%%%%%%%%%%%%%%%%%%%%%%%%%%%%%%%%%%%%%%%%%%%%%%%%%%%%%%%%%%%%%%%%%%%%%%%%%%%%%%
\chapter{Designing and alalizing }
\label{chapter:teo}

%%%%%%%%%%%%%%%%%%%%%%%%%%%%%%%%%%%%%%%%%%%%%%%%%%%%%%%%%%%%%%%%%%%%%%%%%%%%%%%
% Chapter 2: Fundamentos Teóricos 
%%%%%%%%%%%%%%%%%%%%%%%%%%%%%%%%%%%%%%%%%%%%%%%%%%%%%%%%%%%%%%%%%%%%%%%%%%%%%%%

%++++++++++++++++++++++++++++++++++++++++++++++++++++++++++++++++++++++++++++++

En este capítulo se han de presentar los antecedentes teóricos y prácticos que
apoyan el tema objeto de la investigación.

%++++++++++++++++++++++++++++++++++++++++++++++++++++++++++++++++++++++++++++++

\section{Primer apartado del segundo capítulo}
\label{2:sec:1}
  Primer párrafo de la primera sección.

\section{Segundo apartado del segundo capítulo}
\label{2:sec:2}
  Primer párrafo de la segunda sección.


%""""""""""""""""""""""""""""""""""""""""""""""""""""""""""""""""""""""""""""""
\begin{center}
\includegraphics[width=1\textwidth]{images/picn7.eps}\\[0.25cm]
\end{center}
%""""""""""""""""""""""""""""""""""""""""""""""""""""""""""""""""""""""""""""""

%%%%%%%%%%%%%%%%%%%%%%%%%%%%%%%%%%%%%%%%%%%%%%%%%%%%%%%%%%%%%%%%%%%%%%%%%%%%%%%
\chapter{Technics of parallelization}
\label{chapter:exp}

%%%%%%%%%%%%%%%%%%%%%%%%%%%%%%%%%%%%%%%%%%%%%%%%%%%%%%%%%%%%%%%%%%%%%%%%%%%%%%%
% Chapter 3: Procedimiento experimental 
%%%%%%%%%%%%%%%%%%%%%%%%%%%%%%%%%%%%%%%%%%%%%%%%%%%%%%%%%%%%%%%%%%%%%%%%%%%%%%%

Este capítulo ha de contar con seccciones para la descripción de los experimentos 
y del material.
%
También debe haber una sección para los resultados obtenidos y una última de 
análisis de los resultados.

%++++++++++++++++++++++++++++++++++++++++++++++++++++++++++++++++++++++++++++++
\section{Descripción de los experimentos}
\label{3:sec:1}

bla, bla, etc. 

%++++++++++++++++++++++++++++++++++++++++++++++++++++++++++++++++++++++++++++++
\section{Descripción del material}
\label{3:sec:2}

bla, bla, etc. 


%++++++++++++++++++++++++++++++++++++++++++++++++++++++++++++++++++++++++++++++
\section{Resultados obtenidos}
\label{3:sec:3}

bla, bla, etc. 


%------------------------------------------------------------------------------
\begin{figure}[!th]
\begin{center}
\includegraphics[width=0.75\textwidth]{images/figura1.eps}
\caption{Ejemplo de figura}
\label{fig:1}
\end{center}
\end{figure}
%------------------------------------------------------------------------------


%------------------------------------------------------------------------------
\input{tables/table.tex}
%------------------------------------------------------------------------------

%++++++++++++++++++++++++++++++++++++++++++++++++++++++++++++++++++++++++++++++
\section{Análisis de los resultados}
\label{3:sec:4}

bla, bla, etc. 


%""""""""""""""""""""""""""""""""""""""""""""""""""""""""""""""""""""""""""""""
\begin{center}
\includegraphics[width=1\textwidth]{images/picn11.eps}\\[0.25cm]
\end{center}
%""""""""""""""""""""""""""""""""""""""""""""""""""""""""""""""""""""""""""""""
%""""""""""""""""""""""""""""""""""""""""""""""""""""""""""""""""""""""""""""""
\begin{center}
\includegraphics[width=1\textwidth]{images/picn12.eps}\\[0.25cm]
\end{center}
%%%%%%%%%%%%%%%%%%%%%%%%%%%%%%%%%%%%%%%%%%%%%%%%%%%%%%%%%%%%%%%%%%%%%%%%%%%%%%%
\begin{center}
\includegraphics[width=1\textwidth]{images/picn14.eps}\\[0.25cm]
\end{center}
%%%%%%%%%%%%%%%%%%%%%%%%%%%%%%%%%%%%%%%%%%%%%%%%%%%%%%%%%%%%%%%%%%%%%%%%%%%%%%%
\begin{center}
\includegraphics[width=1\textwidth]{images/picn15.eps}\\[0.25cm]
\end{center}
%""""""""""""""""""""""""""""""""""""""""""""""""""""""""""""""""""""""""""""""
%%%%%%%%%%%%%%%%%%%%%%%%%%%%%%%%%%%%%%%%%%%%%%%%%%%%%%%%%%%%%%%%%%%%%%%%%%%%%%%
\chapter{Conclusiones}
\label{chapter:conclusiones}

%%%%%%%%%%%%%%%%%%%%%%%%%%%%%%%%%%%%%%%%%%%%%%%%%%%%%%%%%%%%%%%%%%%%%%%%%%%%%
% Chapter 4: Conclusiones y Trabajos Futuros 
%%%%%%%%%%%%%%%%%%%%%%%%%%%%%%%%%%%%%%%%%%%%%%%%%%%%%%%%%%%%%%%%%%%%%%%%%%%%%%%


According to Toulouse et al. [20], three types of parallelization strategies seem to
be appropriate for the methods most often used in combinatorial optimization: (1)
parallelization of operations within an iteration of the solution method, (2) decomposition
of problem domain or search space, and (3) multi-search threads with various
degrees of synchronization and cooperation. Which of these types is suited for
the parallelization of an optimization method depends mainly on the goal pursued
by the parallelization. Since the enhancement of the solution quality is in the foreground
here, an approach of type 3 is chosen for the parallelization.
An instance of a container loading problem is treated by several processes. Each
process is an instance of the sequential TSA and solves the complete problem. However,
the individual instances are configurated differently. Furthermore, the processes
cooperate through the exchange of calculated solutions. A transmitted solution is
possibly used by the receiving process as a starting point for further search. While
the varying configuration of the processes causes a diversification of the search,
the exchange of solutions serves the intensification of the search within the regions
of best solutions.
Each of the autonomous processes is assigned to a workstation of a local network
(LAN). Hence, more precisely expressed, the parallel TSA is a distributed-parallel
method. It is described more closely in the following.
According to the diversification concept of the sequential TSA, the search in each
process is structured into several phases. In order to determine favourable parameter
settings with respect to the definition of the phases of all processes, a series of experiments
was carried out by means of the sequential TSA (cf. Section 5). The best parameter
settings are distributed approximately evenly over the processes and per
process the most promising parameter settings are, as far as possible, applied in early
phases. In this way the intended intensified exploration of regions that contain solutions
of high quality, is supported to a greater extent.
As to the communication frequency or the number of communications, the type
of the underlying sequential method (here a TSA) is to be considered. The consequence
of a very high communication frequency is that the individual processes
are prevented from intensively exploring limited regions of the search space. Therefore,
a lower communication frequency is to be chosen in advance. Here, an exchange
of best solutions is only carried out at the transition from one phase to the
next phase of each of the processes.
%""""""""""""""""""""""""""""""""""""""""""""""""""""""""""""""""""""""""""""""

%%%%%%%%%%%%%%%%%%%%%%%%%%%%%%%%%%%%%%%%%%%%%%%%%%%%%%%%%%%%%%%%%%%%%%%%%%%%%%%
\thispagestyle{empty}
%\begin{appendix}

%\chapter{Título del Apéndice 1}
%\label{appendix:1}

%\input{tex/apendice1.tex}

%\chapter{Título del Apéndice 2}
%\label{appendix:2}

%\input{tex/apendice2.tex}

%\end{appendix}

%%%%%%%%%%%%%%%%%%%%%%%%%%%%%%%%%%%%%%%%%%%%%%%%%%%%%%%%%%%%%%%%%%%%%%%%%%%%%%%
\addcontentsline{toc}{chapter}{Bibliografía}
\bibliographystyle{plain}


\bibliography{bib/references}

\nocite{*}

%%%%%%%%%%%%%%%%%%%%%%%%%%%%%%%%%%%%%%%%%%%%%%%%%%%%%%%%%%%%%%%%%%%%%%%%%%%%%%%

\end{document}
