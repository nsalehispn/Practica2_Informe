%%%%%%%%%%%%%%%%%%%%%%%%%%%%%%%%%%%%%%%%%%%%%%%%%%%%%%%%%%%%%%%%%%%%%%%%%%%%%%%
% Chapter 3: Procedimiento experimental 
%%%%%%%%%%%%%%%%%%%%%%%%%%%%%%%%%%%%%%%%%%%%%%%%%%%%%%%%%%%%%%%%%%%%%%%%%%%%%%%


For the evaluation of the local arrangements generated for a packing space, two
modes are available which are applied alternatively. The selection of the relevant
mode is also controlled by a parameter, named arrEvalMode.
The first mode, encoded by the parameter value 0, applies a single evaluation criterion:
the total volume of the boxes stowed in the packing space which should be as
large as possible.
The second mode, encoded by the parameter value 1, additionally applies two further
evaluation criteria. These are the already introduced quantities loss volume and
maximum effective volume. Both criteria refer to the residual packing spaces of a local
arrangement. Analogous to the evaluation of subdivisions, the loss volume
should be as small as possible and the maximum effective volume as large as possible.
Since the three evaluation criteria applied are weighted equally, the evaluation procedure
is organized as a series of comparisons of the local arrangements for a packing
space in pairs.
Finally, two additional parameters of the basic heuristic are introduced and
briefly discussed. The parameter maxArr defines the maximum length of the arrangement
list ArrList for a packing space. A local arrangement is only considered in tabu search process if it occurs in ArrList, i.e. belongs to the maxArr best arrangements.
The parameter aboveArr determines whether 2-arrangements of type ‘‘above’’
are generated (parameter value 1) or not (parameter value 0). Like the different
modes for subdivisions and arrangements, the parameter aboveArr serves the diversi-
fication of the tabu searc

%++++++++++++++++++++++++++++++++++++++++++++++++++++++++++++++++++++++++++++++
%\section{Descripción de los experimentos}
%\label{3:sec:1}

%bla, bla, etc. 

%++++++++++++++++++++++++++++++++++++++++++++++++++++++++++++++++++++++++++++++
%\section{Descripción del material}
%\label{3:sec:2}

%bla, bla, etc. 


%++++++++++++++++++++++++++++++++++++++++++++++++++++++++++++++++++++++++++++++
%\section{Resultados obtenidos}
%\label{3:sec:3}

%bla, bla, etc. 


%------------------------------------------------------------------------------
%\begin{figure}[!th]
%\begin{center}
%\includegraphics[width=0.75\textwidth]{images/figura1.eps}
%\caption{Ejemplo de figura}
%\label{fig:1}
%\end{center}
%\end{figure}
%------------------------------------------------------------------------------


%------------------------------------------------------------------------------
%\input{tables/table.tex}
%------------------------------------------------------------------------------

%++++++++++++++++++++++++++++++++++++++++++++++++++++++++++++++++++++++++++++++
%\section{Análisis de los resultados}
%\label{3:sec:4}

%bla, bla, etc. 

