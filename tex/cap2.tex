%%%%%%%%%%%%%%%%%%%%%%%%%%%%%%%%%%%%%%%%%%%%%%%%%%%%%%%%%%%%%%%%%%%%%%%%%%%%%%%
% Chapter 2: Fundamentos Teóricos 
%%%%%%%%%%%%%%%%%%%%%%%%%%%%%%%%%%%%%%%%%%%%%%%%%%%%%%%%%%%%%%%%%%%%%%%%%%%%%%%

%++++++++++++++++++++++++++++++++++++++++++++++++++++++++++++++++++++++++++++++

The structure of local arrangements for a packing space is defined as follows.
A local arrangement consists of one or two so-called blocks and is therefore referred
to as a 1- or 2-arrangement (see Fig. 2). A block is formed from boxes of
the same type. Furthermore, all boxes of a block are arranged in an identical spatial
orientation variant. In each of the three dimensions (x-, y- and z-direction) a block
consists of one or more boxes.
The only block of a 1-arrangement is always placed in the reference corner of the
packing space. From the two blocks of a 2-arrangement, one is arranged in the
reference corner. The second block can alternatively be placed as a neighbour in
x-direction (arrangement type ‘‘in front of’’), as a neighbour in y-direction (arrangement
type ‘‘beside’’) or as a neighbour in z-direction (arrangement type ‘‘above’’). In
the case of a placement according to arrangement type ‘‘in front of’’, the block with
the larger y-dimension is positioned in the reference corner, while for the arrangement
type ‘‘beside’’ the block with the larger x-dimension is positioned in the reference
corner. The arrangement type ‘‘above’’ is only used if both horizontal
dimensions of a block are not smaller than the corresponding dimensions of the
other block. The block with the larger horizontal dimensions is positioned in the reference
corner and the other above. Fig. 2 illustrates a 1-arrangement and two 2-
arrangements of the arrangement types ‘‘in front of’’ and ‘‘beside’’.
For the blocks of an arrangement, the box numbers in all three dimensions are
first determined in such a way that the concerned dimensions of the packing space
are utilized as fully as possible. If the number of boxes of a given type required
for a block exceeds the number of still available items of this type, then the numbers
of boxes are reduced appropriately.
With the selection of a box type and an orientation variant for its block, an 1-
arrangement is defined unambiguously. Analogously a 2-arrangement is completely
defined by the selection of two box types, two orientation variants, and an arrangement
type. All 1-arrangements and all 2-arrangements, which can occur if the box
types, the orientation variants and––in the case of 2-arrangements––the arrangement
type are varied, are generated. However, only those box types for which at least one
item is still available are considered here. Furthermore, the variation of the orientation
variants has to take the orientation constraint (C1) into consideration.

%++++++++++++++++++++++++++++++++++++++++++++++++++++++++++++++++++++++++++++++

%\section{Primer apartado del segundo capítulo}
%\label{2:sec:1}
%  Primer párrafo de la primera sección.

%\section{Segundo apartado del segundo capítulo}
%\label{2:sec:2}
%  Primer párrafo de la segunda sección.

