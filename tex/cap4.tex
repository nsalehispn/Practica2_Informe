%%%%%%%%%%%%%%%%%%%%%%%%%%%%%%%%%%%%%%%%%%%%%%%%%%%%%%%%%%%%%%%%%%%%%%%%%%%%%
% Chapter 4: Conclusiones y Trabajos Futuros 
%%%%%%%%%%%%%%%%%%%%%%%%%%%%%%%%%%%%%%%%%%%%%%%%%%%%%%%%%%%%%%%%%%%%%%%%%%%%%%%


According to Toulouse et al. [20], three types of parallelization strategies seem to
be appropriate for the methods most often used in combinatorial optimization: (1)
parallelization of operations within an iteration of the solution method, (2) decomposition
of problem domain or search space, and (3) multi-search threads with various
degrees of synchronization and cooperation. Which of these types is suited for
the parallelization of an optimization method depends mainly on the goal pursued
by the parallelization. Since the enhancement of the solution quality is in the foreground
here, an approach of type 3 is chosen for the parallelization.
An instance of a container loading problem is treated by several processes. Each
process is an instance of the sequential TSA and solves the complete problem. However,
the individual instances are configurated differently. Furthermore, the processes
cooperate through the exchange of calculated solutions. A transmitted solution is
possibly used by the receiving process as a starting point for further search. While
the varying configuration of the processes causes a diversification of the search,
the exchange of solutions serves the intensification of the search within the regions
of best solutions.
Each of the autonomous processes is assigned to a workstation of a local network
(LAN). Hence, more precisely expressed, the parallel TSA is a distributed-parallel
method. It is described more closely in the following.
According to the diversification concept of the sequential TSA, the search in each
process is structured into several phases. In order to determine favourable parameter
settings with respect to the definition of the phases of all processes, a series of experiments
was carried out by means of the sequential TSA (cf. Section 5). The best parameter
settings are distributed approximately evenly over the processes and per
process the most promising parameter settings are, as far as possible, applied in early
phases. In this way the intended intensified exploration of regions that contain solutions
of high quality, is supported to a greater extent.
As to the communication frequency or the number of communications, the type
of the underlying sequential method (here a TSA) is to be considered. The consequence
of a very high communication frequency is that the individual processes
are prevented from intensively exploring limited regions of the search space. Therefore,
a lower communication frequency is to be chosen in advance. Here, an exchange
of best solutions is only carried out at the transition from one phase to the
next phase of each of the processes.