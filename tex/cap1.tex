%%%%%%%%%%%%%%%%%%%%%%%%%%%%%%%%%%%%%%%%%%%%%%%%%%%%%%%%%%%%%%%%%%%%%%%%%%%%%
% Chapter 1:  
%%%%%%%%%%%%%%%%%%%%%%%%%%%%%%%%%%%%%%%%%%%%%%%%%%%%%%%%%%%%%%%%%%%%%%%%%%%%%%%

By means of the basic heuristic a given container is loaded in several iterations.
Within an iteration a so-called packing space is filled with one or more boxes.
A packing space is an empty rectangular space within the container with defined side
dimensions. In the first iteration the complete interior of the container is used as the
packing space. For the loading of a packing space only box arrangements with a predefined
simple structure are considered. These are called local arrangements. The
feasible local arrangements for a packing space are generated and evaluated by
means of certain criteria. The unused part of the packing space is completely subdivided
into several residual packing spaces. These are filled later. A rough description
of the algorithm of the basic heuristic is given in Fig. 1.
The overall algorithm presented in Fig. 1 requires some comments.

• In order to enhance the chances of loading small packing spaces, the packing
space with the smallest volume is always processed first.

• The container is embedded in a three-dimensional coordinates system. The bottom
left-hand rear corner of a packing space is used as the reference corner.
The coordinates of the reference corner are stored together with the dimensions
of the packing space. The position of a box results from the coordinates of the
reference corner of the respective packing space and its placement within the respective
local arrangement (see below).

• In this section, the basic heuristic is presented as a greedy heuristic. In step (5) the
best evaluated first local arrangement of ArrList is selected. In Section 3 the basic
heuristic is extended in such a way that the best arrangement is not necessarily
used for a packing space with packing space index ipr. Only with this extension
can the basic heuristic be used for the generation of different solutions to a prob-
lem instance. It should be mentioned that an index ipr is only assigned to fillable
packing spaces in which at least one box can be placed.

• At the same time as a local arrangement is generated and evaluated (step 2), the
residual packing spaces that would occur if this local arrangement was used are
generated. In step (6) these residual packing spaces are possibly inserted into
the packing space list PrList.
From the last comment it can be concluded that a more detailed description of the
basic heuristic requires merely a refinement of step (3), which is subsequently described.


%---------------------------------------------------------------------------------
%\section{Sección Uno}
%\label{1:sec:1}
%  Primer párrafo de la primera sección.


%---------------------------------------------------------------------------------
%\section{Sección Dos}
%\label{1:sec:2}
%  Primer párrafo de la segunda sección.

%\begin{itemize}
%  \item Item 1
%  \item Item 2
%  \item Item 3
%\end{itemize}

